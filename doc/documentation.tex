% !TeX spellcheck = en_US
%\documentclass[12pt]{article}

\documentclass[11pt]{report}

\usepackage[a4paper, margin=1in]{geometry}

\usepackage{mathtools}

\usepackage{listings}
\usepackage{xcolor}
\usepackage{mathtools}
\usepackage{pdfpages}
\usepackage[english]{babel}
\usepackage[labelfont=bf]{caption}
\usepackage{float}
\captionsetup{labelfont=bf}
\usepackage[normalem]{ulem}

\useunder{\uline}{\ul}{}

\definecolor{codegreen}{rgb}{0,0.6,0}
\definecolor{codegray}{rgb}{0.5,0.5,0.5}
\definecolor{codepurple}{rgb}{0.58,0,0.82}
\definecolor{backcolour}{rgb}{0.95,0.95,0.95}
\usepackage{titlesec, color}

\definecolor{gray75}{gray}{0.75}
\newcommand{\hsp}{\hspace{10pt}}
%\titleformat{\section}[hang]{\Huge\bfseries}{\thesection\hsp\textcolor{gray75}{|}\hsp}{0pt}{\Huge\bfseries}

\lstdefinestyle{mystyle}{
	backgroundcolor=\color{backcolour},   
	commentstyle=\color{codegreen},
	keywordstyle=\color{blue},
	numberstyle=\tiny\color{codegray},
	stringstyle=\color{orange},
	basicstyle=\ttfamily\footnotesize,
	breakatwhitespace=false,         
	breaklines=true,                 
	captionpos=b,                    
	keepspaces=true,                 
	numbers=left,                    
	numbersep=5pt,                  
	showspaces=false,                
	showstringspaces=false,
	showtabs=false,                  
	tabsize=2
}

\lstset{style=mystyle}

\includeonly{
	chapters/introduction,
	chapters/modeling,
	chapters/implementation,
	chapters/verification,
	chapters/simulation_experiments,
	chapters/conclusions
}

\begin{document}


\begin{titlepage}
	\begin{center}
		\begin{figure}
			\includegraphics[width=\textwidth]{img/marchio_unipi_pant541-eps-converted-to.pdf}         
		\end{figure}
		{\Large
			Computer Engineering\\
			\vspace{5mm} %5mm vertical space
			Foundations of Cybersecurity}\\
		\vspace{30mm} %5mm vertical space
		{\Huge\textbf{\textit{secureCom}}}\\
		\vspace{10mm} %5mm vertical space
		{\Large Group Project Report}\\
		\par\noindent\rule{\textwidth}{0.4pt}
		\begin{flushright}
			\textit{TEAM MEMBERS}:\\ 
			Francesco Iemma\\
			Yuri Mazzuoli\\ 
			Olgerti Xhanej\\
			
		\end{flushright}
		\vfill
		Academic Year: 2020/2021\\        
	\end{center}
\end{titlepage} 
\tableofcontents

\chapter{How To Handle The Chat Request}

\section{General Part}
\noindent Client asks the server to chat with someone: 
\begin{itemize}
	\item The standard channel is used to send either the request for chat and for the server's answer
	\item If the target client accepts then the main process receive the confirmation from the server, then the main process set isChatting to true and it
	starts to chat.
\end{itemize}

\noindent The tricky part deals with the chat request that comes from another client.

\vspace{0.5cm}


\noindent Server asks the client (Alice) if she wants to chat with another client (Bob)

\begin{itemize}
	\item The request must be done through the request channel
	\item The daemon process read isChatting, if it is true the daemon process refuses automatically the server request, otherwise it ask (HANDSHAKE) to the main process if he wants to speak with Bob. 
	\item The main process answer to the server through the daemon tools, if the answer is positive then it sets isChatting to true and it waits for the message from Bob
\end{itemize}


\section{Server Side}
\noindent There are two possibilities:
\begin{enumerate}
	\item Two processes, one for the client main process and another one for the client daemon process. See figure \ref{img: chatRequestProtocolTwoProcesses}.
	
	\begin{figure}[htpb]
		\centering
		\includegraphics[scale=0.5]{img/chatRequestProtocolTwoProcesses.png}
		\caption{Solution with two processes}
		\label {img: chatRequestProtocolTwoProcesses}
	\end{figure}
	
	
	\item One process, it establishes a connection with the main process of the client, then if the server has to sent a chat request to this client it establishes a connection with the daemon process of the client that works as a server process and it is listening on port 4242. See figure \ref{img: chatRequestProtocolOneProcess}.
	
	\begin{figure}[htpb]
		\centering
		\includegraphics[scale=0.5]{img/chatRequestProtocolOneProcess.png}
		\caption{Solution with one process}
		\label {img: chatRequestProtocolOneProcess}
	\end{figure}
	
	\item A third solution can be adding a command to indicate that a client is available to receive chat request. Thus:
	
	\begin{enumerate}
		\item A users is online when it has launched the command "available" (isChatting = false). Notice that in this case the variable isChatting can be called in a more correct way isAvailable.
		
		\item  A users is offline (isChatting = false) if it is not available (if he has not launched the command available or if he is chatting with someone else). In that case offline means also busy whereas online means also available.
	\end{enumerate}

	
\end{enumerate}

\noindent Notes:

\begin{itemize}
	\item ASSUMPTION: who request the chat is the first to send messages
	\item The two processes are connected with the server with two different socket
	\item Server side for each client two socket must be established, one for the standard channel and another one for the request channel
\end{itemize}

\noindent The authentication problem is not present because in the request channel it is used the session key established during the preliminary phase.
\end{document}          
